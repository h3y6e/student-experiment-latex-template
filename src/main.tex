\documentclass[autodetect-engine,dvipdfmx-if-dvi,ja=standard,a4paper,11pt,titlepage]{bxjsarticle}
\usepackage{amssymb,amsmath}
\usepackage{bm}
\usepackage{ascmac}
\usepackage{multirow}
\usepackage{graphicx}
\usepackage{float}
\usepackage[dvipdfmx]{hyperref}

\newcommand{\figref}[1]{図~\ref{#1}}
\newcommand{\tabref}[1]{表~\ref{#1}}
\newcommand{\equref}[1]{式~\eqref{#1}}
\newcommand{\mrm}[1]{\mbox{$_{\mathrm{#1}}$}}
\newcommand{\crm}[1]{\mbox{$\,\mathrm{#1}$}}

\makeatletter
\def\Hline{%
\noalign{\ifnum0=`}\fi\hrule \@height 1.5pt \futurelet
\reserved@a\@xhline}
\makeatother
%===============================================================================
\begin{document}
\title{学生実験のための\LaTeX{}雛形}
\author{京都大学工学部電気電子工学科 \\ 京大太郎}
\maketitle
%===============================================================================
\section{目的}
学生実験のレポート,Wordでもいいけど大学生だし\LaTeX{}で書きたいよね.\\
\LaTeX{}導入のハードルを下げるために雛形を残したいと思った.
%目的ここまで
%===============================================================================
\section{原理}
\subsection{hogeの動作原理}
\label{sub:m_hoge}
{\bf 実験テキストの丸写しは駄目だよ.}
とか言いつつ大体丸写しになる.
%hogeの動作原理ここまで
%原理ここまで
%===============================================================================
\section{方法}
\begin{enumerate}
  \item まずこうするよ
  \item えいやっ
  \item つぎはこうだよ
  \item おーできたパチパチ
\end{enumerate}
%方法ここまで
%===============================================================================
\section{使用器具}
\begin{itemize}
  \item $10\crm{k\Omega}$抵抗とか
  \item オシロスコープ KENWOOD CS-5455
  \item 直流電源装置 KENWOOD PR18-1.2A
  \item その他素子とか
  \item ちゃんと型番も書いてね
\end{itemize}
%使用器具ここまで
%===============================================================================
\section{結果}
図は\figref{fig:5ebec}のように表示する.
\begin{figure}[H]
  \centering
  \includegraphics[width=90mm]{../img/5ebec.jpg}
  \caption{ぼく}
  \label{fig:5ebec}
\end{figure}

図がプロットの場合は\verb+label+を\verb+\label{plt:~}+とする.\\
プロットは\verb+width=90+から\verb+width=110mm+あたりが大体いい感じ.
%結果ここまで
%===============================================================================
\section{考察}

\subsection{数式の書き方}

以下の\equref{eq:sc},\eqref{eq:maxwell}はなんかよく見る式.
\begin{equation}
  I = I\mrm{ph} - I\mrm{d} = I\mrm{ph} - I_0\left\{\exp\left(\frac{qV}{nkT}\right)-1\right\}
  \label{eq:sc}
\end{equation}

\begin{eqnarray}
  \begin{cases}
    \nabla \cdot \bm{B}(t,\bm{x}) &= 0 \\
    \nabla \times \bm{E}(t,\bm{x}) + {\cfrac {\partial \bm{B}(t,\bm{x})}{\partial t}} &= 0 \\
    \nabla \cdot \bm{D}(t,\bm{x}) &= \rho (t,\bm{x}) \\
    \nabla \times \bm{H}(t,\bm{x})-{\cfrac {\partial \bm{D}(t,\bm{x})}{\partial t}} &= \bm{j}(t,\bm{x})
    \label{eq:maxwell}
  \end{cases}
\end{eqnarray}
数式が複数行に渡るなどするときは\verb+eqnarray+を使おう.

\subsection{参考になるサイト}
\label{sub:d_site}
シバニャンさん(@\_6v\_)が書かれたお手軽\LaTeX{} 入門\cite{_6v_}.
僕はこれを見て\LaTeX{} でレポートを書き始めました.\\
\url{http://shiba6v.hatenablog.com/entry/2017/10/26/172825}

\LaTeX{}で書くのが面倒な表はこのサイト\cite{tablegen}を使って書くのがおすすめ.\\
\url{https://www.tablesgenerator.com/}

数式はこのサイト\cite{mathjax}を参考にコピペしていました.\\
\url{http://easy-copy-mathjax.xxxx7.com/}

ちなみに今まで使ってきたハイパーリンクは\verb+hyperref+というパッケージを使うことで設定できるけど,実験レポートは印刷して紙媒体で提出するので必要ない.

%考察ここまで
%===============================================================================
%参考文献
\begin{thebibliography}{9}
\bibitem{text} 京都大学工学部電気系教室 (20xx) 『電気電子工学OO 20xx年度版』←これは書かなくて良いって言う先生もいる
\bibitem{_6v_} 理系大学生のための超手抜きLaTeXレポート入門 \url{http://shiba6v.hatenablog.com/entry/2017/10/26/172825}
\bibitem{tablegen} Table Generator \url{https://www.tablesgenerator.com/}
\bibitem{mathjax} Easy Copy MathJax \url{http://easy-copy-mathjax.xxxx7.com/}
\end{thebibliography}
\end{document}
